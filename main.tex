\documentclass[12pt, a4paper, twoside]{article}
% 12pt: Dimensione del carattere principale.
% a4paper: Dimensione del documento.
% twoside: Impostazione per documenti a due facciate (faccina e retro).

% Pacchetti fondamentali

% Questo pacchetto consente di utilizzare caratteri speciali e lettere accentate direttamente nel tuo documento LaTeX scrivendo in formato UTF-8. È essenziale per la corretta visualizzazione dei caratteri non standard.
\usepackage[utf8]{inputenc}

% Questo pacchetto migliora la resa tipografica dei caratteri, specialmente quelli accentati e speciali, impostando l'encoding dei font a T1. È utile per evitare problemi di visualizzazione e per una migliore qualità della stampa.
\usepackage[T1]{fontenc}

% Adatta automaticamente il documento alle convenzioni linguistiche e tipografiche della lingua specificata (in questo caso, l'italiano). Gestisce correttamente la sillabazione, i nomi dei capitoli, le intestazioni e altre impostazioni linguistiche.
\usepackage[italian]{babel}

% Impostazione delle dimensioni dei margini
\usepackage[
    a4paper,
    top=2.5cm,
    bottom=2.5cm,   % Superiore e inferiore: 2.5 cm
    inner=2.5cm,    % Sinistro: 2 cm margine + 0.5 cm rilegatura
    outer=2cm,      % Destro (outer): 2 cm
    bindingoffset=0.5cm
]{geometry}


% Impostazione dell'interlinea
\usepackage{setspace}
\setstretch{1.15}

% Indentazione della prima riga di ogni paragrafo
\usepackage{indentfirst}

% Pacchetti per elementi aggiuntivi
\usepackage{graphicx}   % per inserire immagini, grafici e figure
\usepackage{hyperref}   % per collegamenti ipertestuali 
\usepackage{amsmath}    % per la scrittura di formule matematiche
\usepackage{booktabs}    % Per tabelle senza linee verticali e doppi
\usepackage{multirow}    % Per celle multicolonna
\usepackage{makeidx}     % Per l'indice
\usepackage{caption}     % Per la formattazione delle didascalie

% Creazione dell'indice
\makeindex

% Formattazione del titolo e dell'autore

% questo importa il pacchetto titling nel tuo documento, permettendoti di utilizzare i comandi \pretitle, \posttitle, \preauthor, e \postauthor per personalizzare la formattazione del titolo e dell'autore.
\usepackage{titling}

% \pretitle{...}: Questo comando definisce il contenuto o la formattazione che precede il titolo quando viene eseguito \maketitle.
% Il titolo è centrato, in corsivo e grandezza \LARGE.
\pretitle{\centering\itshape\LARGE}  

% \posttitle{...}: Questo comando definisce il contenuto o la formattazione che segue il titolo dopo \maketitle.
% \par Assicura che dopo il titolo ci sia uno spazio \vspace adeguato prima del testo successivo.
\posttitle{\par\vspace{0.5cm}}

% \preauthor{...}: Questo comando definisce il contenuto o la formattazione che precede l'autore quando viene eseguito \maketitle.
% Il nome dell'autore sarà visualizzato in maiuscoletto.
\preauthor{\centering\scshape\large}

% \postauthor{...}: Questo comando definisce il contenuto o la formattazione che segue l'autore dopo \maketitle.
\postauthor{\par\vspace{1cm}}

% Definisce il titolo
\title{Titolo della Lezione di \textit{Storia dell'Arte}}
% Definisce l'autore
\author{NOME DELL'AUTORE}

% Inizia il documento
\begin{document}

% Genera il titolo
\maketitle

% Va alla pagina nuova
\newpage

% Crea una sezione (Un capitolo)
\section{Introduzione}
Lorem ipsum dolor sit amet, consectetur adipiscing elit.\footnote{Esempio di nota a piè di pagina.} Sed do eiusmod tempor incididunt ut labore et dolore magna aliqua.

\section{Sezione Principale}
% \cite{fonte} genera un link al riferimento bibliografico. tra parentesi deve esserci lo stesso label che hai assegnato a \bibitem{fonte} nella bibliografia.
Lorem ipsum dolor sit amet, consectetur adipiscing elit. \cite{opera1} Ut enim ad minim veniam, quis nostrud exercitation ullamco laboris nisi ut aliquip ex ea commodo consequat.\footnote{Seconda nota a piè di pagina.}

% Genera una sottosezione (Un paragrafo)
\subsection{Sottosezione}
Duis aute irure dolor in reprehenderit in voluptate velit esse cillum dolore eu fugiat nulla pariatur.\footnote{Terza nota a piè di pagina.} 
% Genera una sottosottosezione (Una sezione del paragrafo)
\subsubsection{Sottosottosezione}
Excepteur sint occaecat cupidatat non proident, sunt in culpa qui officia deserunt mollit anim id est laborum.

% Genera un elenco
% Usa \item[$\square$] per avere i quadrati
% Usa \item[] per non avere nessun simbolo
\begin{itemize}
    \item Primo elemento della lista % Genera un punto della lista
    \item Secondo elemento della lista
    \item Terzo elemento della lista
\end{itemize}

\begin{table}[ht] % Inizio dell'ambiente "table" con opzioni di posizionamento
% Suggerisce a LaTeX di posizionare la tabella here (nel punto corrente) o top (in alto nella pagina). Le opzioni possono includere anche b (bottom), p (pagina dedicata alle tabelle), ecc.
    \centering % Centra la tabella all'interno della pagina
    \caption{Esempio di Tabella} % Didascalia della tabella
    \begin{tabular}{llcc} % Inizio dell'ambiente "tabular" con quattro colonne: due allineate a sinistra (l) e due centrate (c)
        \toprule % Linea orizzontale superiore (fornita dal pacchetto booktabs)
        \multicolumn{2}{c}{Categoria} & \multicolumn{2}{c}{Valori} \\ % Celle che uniscono due colonne ciascuna con intestazioni centrate
        
        % Linee orizzontali parziali sotto le intestazioni
        % \cmidrule(lr){start-end}: Disegna una linea che inizia alla colonna start e finisce alla colonna end.
        % lr: Specifica che le estremità della linea devono essere leggermente abbassate e rialzate per evitare sovrapposizioni con il testo.
        \cmidrule(lr){1-2} \cmidrule(lr){3-4} 
        
        Sottocategoria & Descrizione & Quantità (kg) & Prezzo (€) \\ % Intestazioni delle colonne
        \midrule % Linea orizzontale centrale (fornita dal pacchetto booktabs)

        % Utilizza & per separare il contenuto di ciascuna colonna.
        % Utilizza \\ per terminare la riga.
        A & Descrizione A & 10.50 & 20.00 \\ % Riga di dati
        B & Descrizione B & 5.75 & 15.50 \\ % Riga di dati
        C & Descrizione C & 12.00 & 30.25 \\ % Riga di dati
        
        \bottomrule % Linea orizzontale inferiore (fornita dal pacchetto booktabs)
    \end{tabular} % Fine dell'ambiente "tabular"
    % Etichetta per riferimenti incrociati
    % Permette di fare riferimento alla tabella nel testo utilizzando \ref{tab:esempio}.
    \label{tab:esempio} 
\end{table} % Fine dell'ambiente "table"

Lorem ipsum dolor sit amet, consectetur adipiscing elit. \cite{opera2} Sed do eiusmod tempor incididunt ut labore et dolore magna aliqua. \cite{opera3}

% Creazione di immagini.
% \begin{figure}[ht]
%    \centering
%    \includegraphics[width=0.5\textwidth]{path_immagine}
%    \caption{Descrizione dell'immagine}
%    \label{fig:esempio}
% \end{figure}


\section{Conclusioni}
Lorem ipsum dolor sit amet, consectetur adipiscing elit. Nulla facilisi. 

\newpage

% Crea una sezione non numerata nel documento
\section*{Riferimenti Bibliografici}

% Il comando \addcontentsline serve per aggiungere manualmente una voce a uno dei file di contenuto (come l'indice dei contenuti, la lista delle figure, la lista delle tabelle, ecc.) di un documento LaTeX.
% \addcontentsline{file}{unità}{titolo}
% file: Specifica a quale file di contenuto aggiungere la voce. I file comuni includono:
% toc: Table of Contents (Indice dei Contenuti)
% lof: List of Figures (Lista delle Figure)
% lot: List of Tables (Lista delle Tabelle)
% unità: Definisce il livello gerarchico della voce all'interno del file di contenuto. Per noi i riferimenti bibliografici sono una section.
% titolo: È il testo che apparirà nell'indice. Deve essere inserito tra parentesi graffe {}.
\addcontentsline{toc}{section}{Riferimenti Bibliografici}
\begin{thebibliography}{9}

\bibitem{opera1}
Autore A. \textit{Titolo dell'Opera 1}. Casa Editrice, Anno.

\bibitem{opera2}
Autore B. \textit{Titolo dell'Opera 2}. Casa Editrice, Anno.

\bibitem{opera3}
Autore C. \textit{Titolo dell'Opera 3}. Casa Editrice, Anno.

\end{thebibliography}

\newpage

% Genera l'indice dei contenuti, che include anche i riferimenti bibliografici.
\tableofcontents

\end{document}
